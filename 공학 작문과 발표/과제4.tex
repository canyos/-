\documentclass{article}

\usepackage[hangul]{kotex}
\usepackage{csquotes}
\usepackage{indentfirst}\setlength\parindent{2em}

\title{}
\author{이풍헌\footnote{정보컴퓨터공학과 201924548 stly1974@pusan.ac.kr}}
\date{2022.10.07}

\begin{document}

\maketitle

\section{서론}
코로나 이후 가정에서 즐길 수 있는 엔터테인먼트 수요가 증가하면서 온라인 게임의 인기가 증가했다. 세계 게임 시장은 '크로스 플레이, 게임 구독, IP 게임'등의 트렌트를 통해 약 19.6\% 성장을 보였다.
\footnote{이요훈,"코로나19 이후 게임 산업 급성장…최근 게임 트렌드는?",YTN사이언스,\\2021년 8월 2일 작성.}
지금 출시되고 있는 게임들은 높은 퀄리티를 보인다. 이러한 게임을 만드는데 필수적으로 사용되는 게임엔진을 분석하고 대표적으로 사용되는 UNREAL ENGINE과 unity의 특징을 비교함으로써 게임 개발을 시작할 때 어떤 게임 엔진을 선택해야 하는지 알아보고자 한다.

\section{관련 연구}
\subsection{게임 엔진\protect\footnote{문석환, "게임 개발 도구 '게임 엔진'의 정확한 개념과 그 종류는?",CODING WORLD NEWS,\\2021년 2월 9일 작성.}}
게임 엔진이란 게임을 구동시키는데 필요한 핵심 기능들을 모아놓은 소프트웨어다. 그래픽 엔진, 물리 엔진, 오디오 엔진, UI 시스템, 게임 플레이 프레임워크 등이 포함되어 있어서 개발과정을 단축시켜줄 뿐 만 아니라, 다양한 플랫폼으로의 확장도 제공한다. 


\subsection{게임 시장\protect\footnote{"Engine Launches Each Year",Infogram,\\https://infogram.com/1d560b7e-21a1-437a-91f4-198309bf3e25}}
실제로 우리가 즐기는 수많은 게임은 게임 엔진을 기반으로 개발되었다. 2021년 스팀 게임 기준 UNREAL ENGINE으로 만들어진 게임은 약 119개, unity로 만들어진 게임은 약 342개로 엔진을 특정할 수 있는 게임 중 대략 88\%를 차지한다.
\footnote{가격 \$4.99, 50개 이상의  리뷰 기준}


\end{document}
